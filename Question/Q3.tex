There are 2 key properties of ERP pairs.
1) You cannot know the value that the first measured bit will take, until it is measured.
2) Once you know the value of the first measured bit you know the value of the 2nd bit.

wlog. our EPR pair will be the Bell state $\ket{\phi^+}= \frac{1}{\sqrt{2}}|00> + \frac{1}{\sqrt{2}}|11> $ for some basis $\ket{0}, \ket{1}$.

\begin{proof} 1) You cannot know the value of the first measured bit will take until it is measured.

    If you measure the first bit $q_0$ the $p(q_0=\ket{0}) = \frac{1}{\sqrt{2}}^2 = \frac{1}{2}$, and likewise for $q_1$.
    For both qubits if they are measured first then there is an equal probability of observing $\ket{0}, \ket{1}$.
    Hence you cannot know the value of the first qubit until it is measured.
\end{proof}

\begin{proof} 2) Once you know the value of the first measured bit you know the value of the 2nd bit.

In our system if we know that $q_0=\ket{0}$ then we know the state of the whole system cannot be $\ket{11}$.
There is only one other state with non-zero probability $\ket{00}$, hence we know $q_1=\ket{0}$.
We can make a similar argument if we know the value of $q_1$.

Therefore, once the value of the first qubit is known so to is the value of the second qubit.
\end{proof}




\begin{proof} No-telepathy


    Let us assume that Bob has a function $B$ that processes and measures his qubit $q_b$ to return a guess $x'$ that is always equivalent to Alice's random number $x$.
Alice and Bob cannot communicate so $B$ is only a function of $q_b$.
As $x'$ is deterministic it must also mean $B$ is deterministic where a set of inputs $X_0$ under $B$ map to $x'=0$ and a set of inputs $X_1$ under $B$ map to $x'=1$.
The super position of the entangled pair must include states that are in both $X_0$ and $X_1$ otherwise Bobs output would always be the same and there is no way for Alice to communicate to Bob.

By the first property of ERP pairs, if Bob were to measure $q_b$ first he cannot determine the value of $q_b$ hence $q_b\in X_0 \cup X_1$ and the value of $B$ would be non-deterministic.
Therefore Bob cannot measure first, he must measure after Alice.


By the second property of EPR pairs once the first qubit has been measured the value of the second qubit is known.
So for Alice to communicate with Bob she must influence the state of her qubit $q_a$ so that the state of $q_b$ falls into the correct set for Bob.

The manipulation of $q_a$ to an acceptable state implies it is in a known state or set of states.
If we know the state of a qubit it must be the case we can describe it in the form $\alpha \ket{b_1} + \beta \ket{b_2}$.
However, for qubits to be entangled it must be the case that it is imposible to describe the state of one qubit in isolation.
Hence we have reached a contradiction.
\end{proof}