Let the player's strategy be to guess the first bit if they measure 0 and guess the second bit if they message 1.
\begin{align*}
    P(\text{correct}) &= 1-P(\text{incorrect})\\
        &= 1-(\frac{1}{2}(1-|\beta|^2))\\
        &= 1-\frac{1}{2} + \frac{|\beta|^2}{2}\\
        &= \frac{1}{2} + \frac{|\beta|^2}{2}\\
\end{align*}

Therefore their advantage is 
\[\frac{|\beta|^2}{2} = \Omega(|\beta|^2)\]


\subsection{}
We will call the first bit $Q_0$ and the second bit $Q_1$
Let us offset our measuring basis by $-\frac{\pi}{4}$, we will use $\ket{0'},\ket{1'}$ as new basis vectors where $\ket{0'}$ is the rotation of the $\ket{0}$ basis and likewise for $\ket{1}$.
Let $m$ be the measuring function that returns the probability of measuring a qubit in a given state in our new basis.

\begin{center}
   \newcommand\T{\rule{0pt}{5ex}}       % Top strut
\newcommand\B{\rule[-5ex]{0pt}{0pt}}
\setlength\extrarowheight{3pt}
\begin{tabular}{lll}
    \toprule
Qubit& State&$m(\text{Qubit}, \text{ State})$\\
\midrule
\addlinespace[2ex]$Q_0$&$\ket{0'}$&$\begin{aligned}\phantom{ }&\cos^2(\frac{\pi}{4})\\=&\frac{q}{2}\end{aligned}$\\
\specialrule{0.5pt}{1ex}{1ex}
$Q_0$&$\ket{1'}$&$\begin{aligned}\phantom{ }&\sin^2(\frac{\pi}{4})\\=&\frac{q}{2}\end{aligned}$\\
\specialrule{0.5pt}{1ex}{1ex}
$Q_1$&$\ket{0'}$&$
    \begin{aligned}
        \phantom{ }&\cos^2(\frac{\pi}{4}+\theta)\\
        =&(\cos(\frac{\pi}{4})\cos\theta-\sin(\frac{\pi}{4})\sin\theta)^2\\
        =&\frac{1}{2}(\cos\theta -\sin\theta)^2\\
        =&\frac{1}{2}(\cos^2\theta + \sin^2\theta - 2\cos\theta \sin\theta)\\
        =&\frac{1}{2}(1 - 2\cos\theta \sin\theta)\\
        =&\frac{1}{2} - \cos\theta \sin\theta
    \end{aligned}$\\
\specialrule{0.5pt}{1ex}{1ex}
    $Q_1$&$\ket{1'}$&$
    \begin{aligned}
        \phantom{ }&\sin^2(\frac{\pi}{4}+\theta)\\
        =&(\sin(\frac{\pi}{4})\cos\theta+\cos(\frac{\pi}{4})\sin\theta)^2\\
        =&\frac{1}{2}(\cos\theta +\sin\theta)^2\\
        =&\frac{1}{2}(\cos^2\theta + \sin^2\theta + 2\cos\theta \sin\theta)\\
        =&\frac{1}{2}(1 + 2\cos\theta \sin\theta)\\
        =&\frac{1}{2} + \cos\theta \sin\theta
    \end{aligned}$\\
    \bottomrule
\end{tabular}
\end{center}


Our player will guess $Q_0$ if they measure a $\ket{0'}$ and guess $Q_1$ if they measure a $\ket{1'}$.

\begin{align*}
    p(\text{correct}) &= \frac{1}{2} m(Q_0, \ket{0'}) + \frac{1}{2} m(Q_1, \ket{1'})\\
     &= \frac{1}{2}\cdot \frac{1}{2} + \frac{1}{2} (\frac{1}{2} \cos\theta \sin\theta)\\
     &= \frac{1}{2} + \frac{1}{2} \cos\theta \sin\theta\\
     &= \frac{1}{2} + \frac{1}{2} |\alpha| |\beta|\\
     &= \frac{1}{2} + \frac{1}{2} \sqrt{1-|\beta|^2} |\beta|\\
     &= \frac{1}{2} + \frac{1}{2} \sqrt{|\beta|^2-|\beta|^4}\\
     &= \frac{1}{2} + \frac{1}{2} \Omega(\sqrt{\frac{|\beta|^2}{4}})\\
     &= \frac{1}{2} + \frac{1}{2} \Omega(|\beta|)\\
     &= \frac{1}{2} + \Omega(|\beta|)\\
\end{align*}

Hence the advantage is
\[
    \text{Advantage}=\Omega(|\beta|)
\]



\subsection{}
We will begin by proving the generalized strategy used in the previous questions is the optimal strategy.
We will define $m$ as our measuring function that outputs the probability of measuring a state for a given qubit in our general orthoganal basis $\ket{\alpha}, \ket{\beta}$ 
Let our general strategy be: If we observe the state $\ket{\alpha}$ in our measuring basis guess qubit $Q_a$ otherwise if we observe $\ket{\beta}$ guess qubit $Q_b$ where $a,b\in\{0,1\}$ and $a!=b$, and where $m(Q_a, \ket{\alpha})>m(Q_a, \ket{\beta})$.

We will begin by examining what happens if we measure state $\ket{\alpha}$ on our unknown qubit $Q_?$.
We will let $prob(Q_a|\ket{\alpha})=p$ and $prob(Q_b| \ket{\alpha})=1-p$ where $p>0.5$.
We will use a general probabilistic scheme to make our guess of $Q_a$ or $Q_b$, choosing $Q_a$ with probability $q$ and $Q_b$ with probability $1-q$.
Using our probabilistic selection scheme we will be correct if $Q_?=Q_a$ and we guessed $Q_a$ and if $Q_?=Q_b$ and we guessed $Q_b$.

\begin{align*}
    prob(correct) &= pq + (1-p)(1-q)\\
    &= pq + 1-p-q+pq\\
    &= 2pq-p-q+1
\end{align*}

We are creating an optimal guessing scheme so we consider $p$ to be a constant and look to maximised $prob(correct)$ by changing $q$.

\[
    \arg \max_q(prob(correct)) = \begin{cases}
        q = 0 & \text{if } p<0.5\\
        q = 1 & \text{if } p>0.5\\
    \end{cases}
\]

This result is equivalent to using a greedy guessing strategy.
If $p>0.5$ always guess for action $x$ where $prob(x)=p$, otherwise guess the other action.
Looking at this result in the context of our problem, by definition $p>0.5$ hence we will always guess $Q_a$.

Hence our strategy is optimal...