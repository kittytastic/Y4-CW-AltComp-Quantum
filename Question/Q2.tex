Let the player's strategy be to guess the first bit if they measure 0 and guess the second bit if they message 1.
\begin{align*}
    P(\text{correct}) &= 1-P(\text{incorrect})\\
        &= 1-(\frac{1}{2}(1-\beta^2))\\
        &= 1-\frac{1}{2} + \frac{\beta^2}{2}\\
        &= \frac{1}{2} + \frac{\beta^2}{2}\\
\end{align*}

Therefore their advantage is 
\[\frac{\beta^2}{2} = \Omega(\beta^2)\]


\subsection{}
We will call the first bit $Q_0$ and the second bit $Q_1$
Let us offset out measuring basis by $-\frac{pi}{4}$, we will use $\ket{0'},\ket{1'}$ as new basis vectors where $\ket{0'}$ is the rotation of the $\ket{0}$ basis and likewise for $\ket{y}$.
Let $m$ be the measuring function that returns the probability of measuring a qubit in a given state in our new basis.

%\begin{center}
\begin{tabular}{lll}
    \toprule
Qubit& State&$m(\text{Qubit}, \text{ State})$\\
\midrule
$Q_0$&$\ket{0'}$&$\begin{aligned}\phantom{ }&\cos^2(\frac{\pi}{4})\\=&\frac{q}{2}\end{aligned}$\\
\hline
$Q_0$&$\ket{1'}$&$\begin{aligned}\phantom{ }&\sin^2(\frac{\pi}{4})\\=&\frac{q}{2}\end{aligned}$\\
\hline
$Q_1$&$\ket{0'}$&$
    \begin{aligned}
        \phantom{ }&\cos^2(\frac{\pi}{4}+\theta)\\
        =&(\cos(\frac{\pi}{4})\cos\theta-\sin(\frac{\pi}{4})\sin\theta)^2\\
        =&\frac{1}{2}(\cos\theta -\sin\theta)^2\\
        =&\frac{1}{2}(\cos^2\theta + \sin^2\theta - 2\cos\theta \sin\theta)\\
        =&\frac{1}{2}(1 - 2\cos\theta \sin\theta)\\
        =&\frac{1}{2} - \cos\theta \sin\theta
    \end{aligned}$\\
\hline
    $Q_1$&$\ket{1'}$&$
    \begin{aligned}
        \phantom{ }&\sin^2(\frac{\pi}{4}+\theta)\\
        =&(\sin(\frac{\pi}{4})\cos\theta+\cos(\frac{\pi}{4})\sin\theta)^2\\
        =&\frac{1}{2}(\cos\theta +\sin\theta)^2\\
        =&\frac{1}{2}(\cos^2\theta + \sin^2\theta + 2\cos\theta \sin\theta)\\
        =&\frac{1}{2}(1 + 2\cos\theta \sin\theta)\\
        =&\frac{1}{2} + \cos\theta \sin\theta
    \end{aligned}$\\
    \bottomrule
\end{tabular}
%\end{center}


Our player will guess $Q_0$ if they measure a $\ket{0'}$ and guess $Q_1$ if they measure a $\ket{1'}$.

\begin{align*}
    p(\text{correct}) &= \frac{1}{2} m(Q_0, \ket{0'}) + \frac{1}{2} m(Q_1, \ket{1'})\\
     &= \frac{1}{2}\cdot \frac{1}{2} + \frac{1}{2} (\frac{1}{2} \cos\theta \sin\theta)\\
     &= \frac{1}{2} + \frac{1}{2} \cos\theta \sin\theta\\
     &= \frac{1}{2} + \frac{1}{2} \alpha \beta\\
     &= \frac{1}{2} + \frac{1}{2} \sqrt{1-\beta^2} \beta\\
     &= \frac{1}{2} + \frac{1}{2} \sqrt{\beta^2-\beta^4}\\
     &= \frac{1}{2} + \frac{1}{2} \Omega(\sqrt{\frac{\beta^2}{4}})\\
     &= \frac{1}{2} + \frac{1}{2} \Omega(\beta)\\
     &= \frac{1}{2} + \Omega(\beta)\\
\end{align*}

Hence the advantage is
\[
    \Omega(\beta)
\]