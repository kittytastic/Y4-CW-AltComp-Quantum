\subsection{}
Let the player's strategy be to guess the first bit if they measure 0 and guess the second bit if they message 1.
\begin{align*}
    P(\text{correct}) &= 1-P(\text{incorrect})\\
        &= 1-(\frac{1}{2}(1-|\beta|^2))\\
        &= 1-\frac{1}{2} + \frac{|\beta|^2}{2}\\
        &= \frac{1}{2} + \frac{|\beta|^2}{2}\\
\end{align*}

Therefore their advantage is 
\[\frac{|\beta|^2}{2} = \Omega(|\beta|^2)\]


\subsection{}
We will call the first bit $Q_0$ and the second bit $Q_1$
Let us offset our measuring basis by $-\frac{\pi}{4}$, we will use $\ket{0'},\ket{1'}$ as new basis vectors where $\ket{0'}$ is the rotation of the $\ket{0}$ basis and likewise for $\ket{1}$.
Let $m$ be the measuring function that returns the probability of measuring a qubit in a given state in our new basis.

\begin{center}
   \newcommand\T{\rule{0pt}{5ex}}       % Top strut
\newcommand\B{\rule[-5ex]{0pt}{0pt}}
\setlength\extrarowheight{3pt}
\begin{tabular}{lll}
    \toprule
Qubit& State&$m(\text{Qubit}, \text{ State})$\\
\midrule
\addlinespace[2ex]$Q_0$&$\ket{0'}$&$\begin{aligned}\phantom{ }&\cos^2(\frac{\pi}{4})\\=&\frac{q}{2}\end{aligned}$\\
\specialrule{0.5pt}{1ex}{1ex}
$Q_0$&$\ket{1'}$&$\begin{aligned}\phantom{ }&\sin^2(\frac{\pi}{4})\\=&\frac{q}{2}\end{aligned}$\\
\specialrule{0.5pt}{1ex}{1ex}
$Q_1$&$\ket{0'}$&$
    \begin{aligned}
        \phantom{ }&\cos^2(\frac{\pi}{4}+\theta)\\
        =&(\cos(\frac{\pi}{4})\cos\theta-\sin(\frac{\pi}{4})\sin\theta)^2\\
        =&\frac{1}{2}(\cos\theta -\sin\theta)^2\\
        =&\frac{1}{2}(\cos^2\theta + \sin^2\theta - 2\cos\theta \sin\theta)\\
        =&\frac{1}{2}(1 - 2\cos\theta \sin\theta)\\
        =&\frac{1}{2} - \cos\theta \sin\theta
    \end{aligned}$\\
\specialrule{0.5pt}{1ex}{1ex}
    $Q_1$&$\ket{1'}$&$
    \begin{aligned}
        \phantom{ }&\sin^2(\frac{\pi}{4}+\theta)\\
        =&(\sin(\frac{\pi}{4})\cos\theta+\cos(\frac{\pi}{4})\sin\theta)^2\\
        =&\frac{1}{2}(\cos\theta +\sin\theta)^2\\
        =&\frac{1}{2}(\cos^2\theta + \sin^2\theta + 2\cos\theta \sin\theta)\\
        =&\frac{1}{2}(1 + 2\cos\theta \sin\theta)\\
        =&\frac{1}{2} + \cos\theta \sin\theta
    \end{aligned}$\\
    \bottomrule
\end{tabular}
\end{center}


Our player will guess $Q_0$ if they measure a $\ket{0'}$ and guess $Q_1$ if they measure a $\ket{1'}$.

\begin{align*}
    p(\text{correct}) &= \frac{1}{2} m(Q_0, \ket{0'}) + \frac{1}{2} m(Q_1, \ket{1'})\\
     &= \frac{1}{2}\cdot \frac{1}{2} + \frac{1}{2} (\frac{1}{2} \cos\theta \sin\theta)\\
     &= \frac{1}{2} + \frac{1}{2} \cos\theta \sin\theta\\
     &= \frac{1}{2} + \frac{1}{2} |\alpha| |\beta|\\
     &= \frac{1}{2} + \frac{1}{2} \sqrt{1-|\beta|^2} |\beta|\\
     &= \frac{1}{2} + \frac{1}{2} \sqrt{|\beta|^2-|\beta|^4}\\
     &= \frac{1}{2} + \frac{1}{2} \Omega(\sqrt{\frac{|\beta|^2}{4}})\\
     &= \frac{1}{2} + \frac{1}{2} \Omega(|\beta|)\\
     &= \frac{1}{2} + \Omega(|\beta|)\\
\end{align*}

Hence the advantage is
\[
    \text{Advantage}=\Omega(|\beta|)
\]



\subsection{}
Firstly, we will continue the use of $\ket{0'}$ and $ket{1'}$ as basis vectors where $\ket{0'}$ is a rotation of $\ket{0}$ by $-\delta$ radians, and likewise of $\ket{1'}$ and $\ket{1}$.
wlog $0 \leq \Delta \leq \frac{\pi}{4}$.
In the context the 2 qubits $Q_0$ and $Q_1$ allowing $\Delta>\frac{\pi}{4}$ will only generate rotationally symmetrical solutions to the solution in the range $0 \leq \Delta \leq \frac{\pi}{4}$.
Therefore this restriction can be made wlog.
Let $m$ be the measuring function that outputs the probability of measuring a state for a given qubit the basis $\ket{0'}, \ket{1'}$ 

\begin{proof} The greedy guessing strategy used in 2a and 2b is the optimal strategy.
Let the greedy strategy be: When we measure the unknown qubit $Q_?$, if we observe the state $\ket{0'}$ guess qubit $Q_0$ otherwise if we observe $\ket{1'}$ guess qubit $Q_1$.

We will look at the cases where measuring $Q_?$ gives state $\ket{1'}$.
The same argument can be applied to the case where $\ket{0'}$ is measured.
Thanks to our restriction of $\Delta$, we know that $m(Q_1, \ket{1'})\geq m(Q_0, \ket{1'})$ and also that $p(Q_1|\ket{1'}) \geq p(Q_0|ket{1'})$.
Let $p(Q? = Q_1) = p(Q_1|\ket{1'}) = x$ and $p(Q? = Q_0) = p(Q_0|\ket{1'}) = 1-x$.

Let us assume that there is a better probabilistic scheme to make our guess.
We will guess $Q_1$ with probability $q$ and $Q_0$ with probability $1-q$.
Using our probabilistic selection scheme we will be correct if $Q_?=Q_0$ and we guessed $Q_0$ and if $Q_?=Q_1$ and we guessed $Q_1$.

\begin{align*}
    p(correct) &= xq + (1-x)(1-q)\\
    &= xq + 1-x-q+xq\\
    &= 2xq-x-q+1
\end{align*}

Our scheme will only has control over the probability of $q$, we consider $x$ to be a constant.

\[
    \arg \max_q(p(correct)) = \begin{cases}
        q = 0 & \text{if } x<0.5\\
        q = 1 & \text{if } x\geq 0.5\\
    \end{cases}
\]

This result is equivalent to using a greedy guessing strategy.
If $x\geq 0.5$ always guess action $Y$ where $prob(Y)=x$, otherwise guess the other action.
In this case we know $x\geq 0.5$ and that the action $Y$ is $Q_? = Q_1$.

Therefore, we have reached a contradiction as our probabilistic strategy has decayed into a greedy strategy.
Thus, the greedy strategy must be optimal. 
\end{proof}

As we are already using an optimal strategy to select a qubit on measurement, the only other option to improve the advantage is to change the measurement basis by adjusting $\Delta$.

The probability of guessing correctly is 
\begin{align*}
    p(correct) &= p(c=0 \text{ and } \text{ measure } \ket{0'}) + p(c=1 \text{ and } \text{ measure } \ket{1'})\\
    &=\frac{1}{2} \cos^2 (\Delta) + \frac{1}{2}\sin^2(\Delta + \theta) 
\end{align*}
where $\theta$ is the angle between $Q_0$ and $Q_1$.

To find the maximum of $p(correct)$ with respect to $\Delta$ we will find stationary points
\begin{align*}
    \frac{\partial }{\partial \Delta} = \sin (\Delta + \theta) \cos (\Delta + \theta) - \sin \Delta \cos \Delta
\end{align*}
Looking at the graph $p(correct)$ for any value of $0\leq\theta\leq\frac{\pi}{2}$ shows there is only 1 stationary point in the range $0\leq \Delta \leq \frac{\pi}{4}$ and that is a maximum.
Evaluating the derivative at $\Delta = \frac{\pi}{4} - \frac{\theta}{2}$ we find:
\begin{align*}
    &\frac{\partial }{\partial \Delta}\Bigr\rvert_{\Delta = \frac{\pi}{4} - \frac{\theta}{2}}\\
    =& \sin (\frac{\pi}{4} + \frac{\theta}{2} ) \cos (\frac{\pi}{4} + \frac{\theta}{2} ) - \sin(\frac{\pi}{4} - \frac{\theta}{2}) \cos(\frac{\pi}{4} - \frac{\theta}{2})\\
    =& \frac{1}{2}\sin (\frac{\pi}{2} + \theta ) - \frac{1}{2}\sin(\frac{\pi}{2} - \theta)\\
    =& \frac{1}{2}\cos (\theta ) - \frac{1}{2}\cos(- \theta)\\
    =& \frac{1}{2}\cos (\theta ) - \frac{1}{2}\cos( \theta)\\
    =& 0
\end{align*}
Hence $\Delta = \frac{\pi}{4} - \frac{\theta}{2}$ is the stationary point which is a maximum.

In 2b we used the optimal strategy with the optimal value of $\Delta$ to get an advantage of $\Omega(|\beta|)$.
Therefore, an advantage of $\Omega(|\beta|)$ must be optimal and cannot be improved upon.
